\chapter{Concluding Remarks}\label{concluding_remarks}

The purpose of this report was initially presented as three correlated objectives. The first objective has been to gain insight into the e-scooter industry and how existing literature can be applicable for optimization problems regarding e-scooter sharing systems. Through a review of relevant literature on the topic, it has been revealed that an optimal allocation and routing of service vehicles operating in an e-scooter sharing system is not covered in the literature. However, some of the fundamental aspects of the problem overlap with problems presented in previous studies. Routing problems for bike-sharing systems and free-floating systems share similar characteristics with the Static E-scooter Battery Swap Rebalancing Problem (SEBSRP). The most important distinction that motivates this report is the combination of battery swaps and rebalancing moves in ESS, in addition to the characteristic that not all e-scooters have to be visited during the duration of a shift.

The second objective of this report has been to formulate and implement a mathematical model handling the daily operation of the e-scooter operators. Thus, two mathematical models have been formulated and implemented to solve the SEBSRP. Both models aim to increase the overall availability of a fleet of e-scooters by determining optimal routes for service vehicles performing battery swaps and rebalancing moves. The models have been formulated as variants of a Team Orienteering Problem, where a reward is given for every visit to a location, where either a battery swap or a rebalancing move is performed. Furthermore, the models have been tested on real-world data retrieved through the API of EnTur.no, which has been processed to fit the scope of this report and the limitations of the solution method.  

The third and final objective has been to analyze the impact different parameters have on the model and how it can provide value for the operators.  Thus, a computational study analyzing the results of running the model on different test instances has been conducted.  The models formulated are not yet able to solve instances of realistic size, but nevertheless, the computational study has provided valuable insights to the problem of operating an ESS. Initially, it has been discovered that the solution time for solving the SEBSRP grows exponentially as more e-scooters are added to an instance. Additionally, the solution time is sensitive to the number of service vehicles used in the problem and the duration of their shifts. However, relaxing certain integer and binary constraints in the model improves the computational time considerably. Furthermore, the solutions from the two models provide useful results from an economic point of view. By running the models on smaller test instances, the models are solved to optimality. The resulting solutions give an indication of how decisions regarding the shift duration, the number of service vehicles and vehicle routing can be made optimally. Future research upscaling the size of the instances to replicate real-world scenarios should be done if the model is to be able to provide decision support for the operators of ESS. This could improve the operational costs of the operators significantly and is discussed further in Chapter \ref{future_research}.

The research presented in this report has provided an insight into ESS and the dynamics of the e-scooter industry. These insights will serve as a foundation for the future work on a master thesis on the same topic. ESS have experienced increasing popularity over the past years, and will probably emerge in continually growing numbers all over the world in the future. As a result, further scientific research into how they can be operated could be beneficial, both for the operators and from a socio-economic point of view. Although the models implemented in this report are not able to solve instances of real-world systems, they give an insight into how operational decisions in the e-scooter industry can be made, and how the operators can use optimization research to increase the overall availability of their e-scooter fleet and to decrease their operational costs. 

