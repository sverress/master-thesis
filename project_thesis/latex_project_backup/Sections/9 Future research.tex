\chapter{Future Research}\label{future_research}

In this section, the possibilities of future research are presented. In \Cref{solving larger instances} a presentation of possible solution methods to improve the size of solvable instances is discussed. Further, a description of the possibility of a proper demand analysis for setting the input arguments of the model is presented in \Cref{demand analysis}. Lastly, some extensions that could be added to the model are given in \Cref{extensions of the current model}. 

\section{Solving Larger Instances}\label{solving larger instances}

The models presented in \cref{math_model} are able to solve instances of a size around 20-30 e-scooters. However, as operators normally have a couple of thousand e-scooters, the models are not usable for practical applications. Future research could include solutions methods to improve the size of solvable instances. 

Both column generation and a branch and price scheme could be possible exact solution methods. These are implemented by adding improving routes to a master problem until there are no improving routes to be added and are often used in vehicle routing problems. Moreover, heuristic methods can also be used. A tabu neighborhood search algorithm could be efficient, as well as a genetic algorithm. These methods might be good approaches as they have been known to solve the TOP efficiently, as discussed in \Cref{TOP}. Additionally, heuristic methods are often able to solve problems faster than the exact methods which can be necessary for this domain.

\section{Demand Analysis}\label{demand analysis}
There has not been any consideration of demand analysis in this report. Hence, a natural addition will be a proper analysis of the demand in order to give realistic input arguments to the model. In this analysis, a better way to determine the geographical boundaries of zones can be looked at. With these zones, predicting the ideal state for each of them could be beneficial. This is a difficult problem to solve as the ideal state is hard to verify the correctness of. However, e-scooter companies have data on how many times a user opens the app without using an e-scooter. As this can be interpreted as unmet demand, some analysis can be done to approximate actual demand. Moreover, the reward function of the alternative model could be fitted more to the supply and demand of the zones to better represent the added availability. 

\section{Extensions of the Current Model}\label{extensions of the current model}
There are several different aspects to be added to make the formulation of the current model closer to the real-world problem. Introducing different kinds of vehicles is one of them. Operators today use bikes as well as cars for swapping batteries. The bikes are only able to perform battery swaps. Additionally, a clustering algorithm to combine e-scooters in close proximity into a single location can reduce the problem complexity. The model would then have to handle multiple e-scooters in a single location, being able to prioritize locations where a lot of availability can be added within a small radius.

The current problem only considers operations during the night. However, the e-scooter companies also operate during the day while the e-scooters are moving. In this scenario, a dynamic model should be used. Such a model can be solved every hour giving the service vehicles the routes with the highest expected added availability. To make this problem dynamic, an analysis of the possible demand scenarios and their probabilities should be done to determine where e-scooters should be moved to maximize availability.

The current model uses battery percentage as a measure of the availability added to the system. There is a possibility of using other measures as well. One could minimize the time a certain zone is under a given availability limit. This method would cut away zones being too far away from a respectable deviation from the ideal state. Alternatively, minimizing lost trips in a zone due to the lack of availability can also be an efficient measure. Other interesting measures of availability could also exist and be explored further.

As stated in the problem description, there are three ways for an e-scooter to be unavailable, lack of battery percentage, no customer near it, or the event of an e-scooter being damaged. In this report, the consideration of damaged e-scooters is ignored. However, this is a big part of the operations for the e-scooter companies and something that could be added in future research. 

