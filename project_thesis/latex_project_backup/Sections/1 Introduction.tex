\chapter{Introduction} \chaptermark{Introduction} \label{introduction}
\nocite{imt_software_wiki}  % This is an example of how to add a reference to the bibliography at the end without having it displayed as a reference within the text

Today, more than half of the world’s population lives in urban areas. However, this proportion is expected to increase to approximately 70\% by 2050 \citep{nations_68_2018}. With global urbanization on the rise, traffic congestion and pollutant emissions rises accordingly. Consequently, the use of green methods of transportation, like bicycles and electric micromobility vehicles, has gained increasing popularity. This has led to a growing amount of sharing systems, providing a substitute to historically popular and less environmentally friendly means of transportation. These sharing systems include both bicycles, electrical scooters and cars, and reduces the traffic congestion caused by the growing urbanization. 

This report examines different kinds of sharing systems, but focuses on the systems designed for electrical kick scooters (e-scooters). The concept of e-scooter sharing systems (ESS) is based on operators providing e-scooters for customers to use as transportation in urban areas, in exchange for a payment. To use the system, the customers locate an e-scooter, unlock it through a mobile application, travel to their destination, and lock the e-scooter after use. The ESS are aimed at being seamless substitutes to traditional transportation methods and provide effective and cheap possibilities for short-distance travel.

The operators are responsible for maintaining the availability of their fleet of e-scooters. In modern ESS, this is done through three sets of operations. Firstly, the operators ensure that each e-scooter has a significant battery capacity, by changing their batteries. Secondly, the operators consider the future demand for e-scooters and actively allocate them accordingly. Lastly, the fleet of e-scooters is in constant need of technical maintenance, forcing the operators to replace and perform service on the e-scooters regularly. 

Everyday operations are costly for the operators and are responsible for a major part of their variable costs. Thus, ensuring an optimal way of performing the operations is of great economic value to them. As the ESS become more common and larger in size, the need for a decision support system handling the operations could differentiate the companies prospering and those going out of business. Additionally, the availability of the e-scooters is an important area for e-scooter operators to differentiate themselves from their competitors. In a market as saturated as the ESS market, the availability of the operators e-scooter fleet could provide the advantage necessary to acquiring a satisfactory market share and becoming a profitable company.

The purpose of this report consists of three objectives; 1) to gain insight into the e-scooter industry and how existing literature can be applicable for optimization problems regarding e-scooter sharing systems, 2) to formulate and implement a mathematical model handling the daily operation of the e-scooter operators, and 3) to analyze the impact different parameters has on the model and how it can provide value for the operators. Altogether, this will provide an understanding of the mathematical model and its contributions to maximizing the operational decisions of the operators. This will in turn provide a solid foundation for further research on the topic.

This report starts off with an introductory description of the concepts of micromobility, sharing systems, e-scooters, and the challenges of ESS, found in \Cref{background}. This is followed by a literature study on relevant material in Chapter \ref{lit_study}. The literature study focuses on how this report contributes to the existing research on the topic, as well as how it differentiates itself from articles previously written. As there has not been conducted a lot of research on ESS, literature on similar sharing systems is also discussed. Additionally, mathematical models formulated to optimize the sharing systems are presented.

Further, a description of the problem discussed in this report is presented in Chapter \ref{scope and problem description}, before a mathematical model aimed at solving the presented problem is formulated in Chapter \ref{math_model}. The mathematical model makes use of multiple sets and parameters, which are based on real-world data. Thus, an explanation of the retrieval and processing of the data used to test the mathematical model is found in Chapter \ref{case_study}. This is followed by a computational study in Chapter \ref{comp_study}. This study elaborates on the technical and economical observations from the results retrieved from the model. In addition, a comparison of the performance of different alterations of the model is presented in this chapter. To round off, Chapter \ref{concluding_remarks} provides concluding remarks for the report, before possible areas of future research are discussed in Chapter \ref{future_research}. Altogether, the report provides a general understanding of the dynamics of the e-scooter industry, the business of the operators, and how optimization research could be of great significance when performing operations on the fleet of e-scooters. 

